\section{Trabalhos Relacionados}\label{relatedworks}

%The Optimized Exchange Protocol (IEEE 11073-20601) is the core of X73-PHD family. It defines the communication syntax in the Domain Information Model (DIM), machine states and services types in the Service Model and procedures in the Communication Model.
A parte IEEE 11073-20601 Optimized Exchange Protocol é o principal componente da família X73-PHD. Ele define a sintaxe de comunicação usando um Modelo de Domínio de Informação (Domain Information Model DIM), máquinas de estados e tipos de serviços de comunicação no Modelo de Serviço e procedimentos no Modelo de Comunicação

%The Domain Information Model (DIM) defines all common classes and data types used by device types. These classes are expanded by the specialization profiles according to the needs of each device. The Service Model defines the types of messages that can be exchanged between an agent and a manager and the conceptual context in which they are being transmitted \cite{b17}. The communication model defines the procedures to be followed under a normal operation, an exit condition, or when an error occurs.
O DIM define todas as classes e tipos de dados usados pelos dispositivos. Estas classes podem ser expandidas pelos agentes de acordo com a necessidade de cada um. O Modelo de Serviço define os tipos de mensagens que podem ser trocadas entre um agente e um gerente e o contexto na qual essas mensagens estão sendo transmitidas \cite{b17}. Por fim, o Modelo de Comunicação define os procedimentos a serem seguidos sob condições normais ou quando um erro ocorre.

%The 11073 family of standards includes specialization profiles, that is, each agent has an associated standard that describes its data representation. For example, the standard 11073-10408 sets standards for a thermometer, and the 11073-10415 for a balance. These specialization defines the DIM of each device, its attributes, methods, and events of each agent class.
A família de padrões 11073 oferece perfis de especialização, isto é, cada agente possui suas próprias normas, as quais descrevem sua representação de dados, atributos, métodos e eventos. Por exemplo, o perfil IEEE 11073-10408 define as classes e objetos para um termômetro, já o perfil IEEE 11073-10415 faz o mesmo para uma balança. 

%Antidote Stack or Antidote Library is an implementation of the Optimized Exchange Protocol (IEEE 11073-20601) developed by Signove as part of the SigHealth Platform \footnote{SigHealth is a platform for remote patient monitoring and data management using personal wireless devices for health.}. This library is the first open source implementation of this standard, and was developed in ANSI-C with modular architecture, which allows code portability for different platforms.
A biblioteca Antidote é uma implementação da parte IEEE 11073-20601 \textit{Optimized Exchange Protocol}, desenvolvida pela empresa Signove, como parte da plataforma SigHealth\footnote{SigHealth é uma plataforma de monitoramento remoto de pacientes e gestão de dados utilizando dispositivos pessoais sem fio para a saúde.}. Esta biblioteca é a primeira implementação livre e de código aberto deste padrão, e foi desenvolvida em ANSI-C com arquitetura modular, o qual permite portabilidade de código para diferentes plataformas.

%With the popularization of the 11073 standard, several improvements were proposed to enhance the operation and interoperability. %In \cite{b7},\cite{b8} and \cite{b9} the integration of X73-PHD and IoT protocols, such as MQTT and COaP, are proposed to be used as transport protocols, enabling personal health devices to share health information directly through the Internet, using low power consumption and few control messages. Those works also discuss the availability of enabling IoT technologies for health information as well as the mapping of messages from X73-PHD into IoT protocols. All those works used real devices with Antidote as their application layer protocol.
Com a popularização da família de padrões 11073, vários aperfeiçoamentos e integrações com outros sistemas foram feitos. Nos trabalhos \cite{b7}, \cite{b8} e \cite{b9}, é proposta a integração do padrão X73-PHD com protocolos de IoT, como MQTT e CoAP, para serem usados na camada de transporte de PHDs. Esses protocolos auxiliariam o compartilhamento de informação de saúde diretamente na Internet, usando baixo consumo energético e poucos pacotes de controle. Nesse trabalho, também é apresentado o mapeamento das mensagens X73-PHD para mensagens MQTT e CoAP e a biblioteca Antidote é utilizada na camada de aplicação.   

%Another project on 11073 standards is \cite{b11}, which has developed an interoperable end-to-end remote patient monitoring platform using ZigBee Health Care Profile as transport layer and a Machine to Machine (M2M) solution to provide wide area network connectivity. That work also includes a web application on the clinical side (server side) and use the standards and frameworks provided by Integrating the Healthcare Enterprise (IHE) \cite{b13} and Health Level Seven (HL7) \cite{b12} to ensure end-to-end interoperability. Those two companies advocate a world in which everyone can securely access and use the right health data when and where they need it.
O padrão X73-PHD é aplicada em várias áreas da saúde e de e-Health, como por exemplo, diagnóstico precoce de doenças crônicas e  monitoramento remoto de pacientes em suas residências. No trabalho \cite{b22}, os autores propõem uma coleção de dispositivos vestíveis X73-PHD, \textit{plug-and-play}, que coletam continuamente, dados fisiológicos e do ambiente. Neste projeto, três dispositivos são utilizado: dispositivos sensores, que coletam informações fisiológicas e do ambiente; Os \textit{Data Loggers}, que são dispositivos móveis, que armazenam temporariamente os dados dos sensores; e por fim, as \textit{Base Stations}, terminais que recebem, processam, armazenam e exibem informações recebidas do \textit{data loggers}. Como camada de transporte os autores, utilizaram o Bluetooth e, na camada de aplicação, o padrão X73-PHD.  

%The X73-PoC version provides a mechanism to remotely control agents. This mechanism is defined in the standards X73-10201 and X73-20301. However, the X73-PHD has no mechanisms to do such a thing. So, in \cite{b14} the author proposes to adapt the remote control capabilities from X73-PoC to X73-PHD with an acceptable overhead and no extra cost to manufacturers. This mechanism has to be installed in manager and agents, and the remote control consists of being able to change the units of measurements (e.g. - changing from kilograms to pounds) directly from the manager, a smartphone or a computer engine in nursing units.
Outro trabalho interessante é o desenvolvimento de um adaptador para dispositivos que não seguem o padrão X73-PHD. Como mostrado em \cite{b23}, os autores sugerem um dispositivo intermediário entre os sensores não padronizados e o gerente. Este dispositivo faz a tradução das mensagens vindo desses sensores para mensagens X73-PHD, e então transmite os dados traduzidos para um gerente padronizado. Este trabalho não deixa claro como foram feitos o mapeamento das mensagens e em quais dispositivos o adaptador proposto funciona. 

%In this work, we use the Antidote library to run the agent and manager X73-PHD stack over Castalia. Five different device profiles are used: 11073-10406 Basic electrocardiograph (1 to 3 lead ECG), 11073-10404 Pulse oximeter, 11073-10408 Thermometer, 11073-10417 Glucose meter, and 11073-10407 Blood pressure monitor. All these agents are used to represent body sensors in WBAN simulations. To fulfill that purpose, the X73-PHD reliable data transfer mechanism was adjusted to work on a WBAN scenario, a wireless environment, with a faulty media, and lacking transport layer services.
Neste artigo, usamos a biblioteca Antidote para propor uma camada de aplicação, seguindo o padrão X73-PHD no simulador Castalia. Cinco perfis de especialização diferentes são utilizados: 11073-10406 electrocardiograma básico de 3 derivações, 11073-10404 oxímetro de pulso, 11073-10408 termômetro, 11073-10417 medidor de glicose e 11073-10407 medidor de pressão sanguínea. Além disso, um novo modo de comunicação é proposto como extensão ao padrão para uso em WBANs, como será detalhado na próxima seção.

%VINICIUS - Eu tirei pra deixar o capítulo todo mais focado em PHD%
%In \cite{b6} the authors presents an open-source energy-harvesting simulation framework called GreenCastalia which supports multi-source and multi-storage energy harvesting architectures developed for the Castalia simulator. GreenCastalia project focus is to simulate realistic battery discharge on devices. The majors modifications to the original Castalia code was made to the Resource Manager module. Also in this module the authors adds the energy-harvesting systems that provides a more realistic battery model.