\section{Conclusão}\label{conclusion}
%In this paper, we have presented a proposal to simulate personal health devices in Castalia Simulator. Our application layer follows the X73-PHD standard with the aid of the Antidote Library. Five agents were implemented, and they simulate real personal health devices. In adittion, a new reliable data transfer mode was proposed, the retransmission mode, to adjust the X73-PHD protocol to WBAN scenarios, where a reliable transport layer is usually not available. The retransmission mode aimed at reducing the number of disassociation/reassociations that take place after a message or acknowledgement is lost.
Neste artigo, foi apresentada uma extensão ao protocolo X73-PHD que pode ser testado no Simulador Castalia e, adaptou dispositivos X73-PHD para serem empregados em cenários WBAN.
Como ilustrado nos resultados, esta proposta melhora o desempenho dos PHDs, ao mesmo tempo que oferece um mecanismo de entrega confiável. 

%As future work, we intend to develop a manager-initiated transmission, where the user can set the amout of measurements each agent must collect. The codes for Castalia Application and for Antidote Modified Library can be found respectively at \url{https://github.com/conqlima/Antidote} and \url{https://github.com/conqlima/11073PhdApplication}.
Como trabalhos futuros, será desenvolvida a função que permite ao usuário especificar, no gerente, a quantidade de dados que deve ser transmitida por cada agente. Os códigos da Aplicação do Castalia e da biblioteca Antidote modificada, podem ser encontrados nos endereços: \url{https://github.com/conqlima/Antidote} e \url{https://github.com/conqlima/11073PhdApplication}. 
%The results shows the advantages of using the retransmission mode over the standards' confirmed mode, increasing the message delivery ratio and reducing the overhead of the protocol. 

